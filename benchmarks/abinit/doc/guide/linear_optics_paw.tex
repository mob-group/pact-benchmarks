%\documentclass[prb,byrevtex]{revtex4-1}
\documentclass{article}

\usepackage{graphicx}
\usepackage{amssymb}
%%\usepackage[version=3]{mhchem}

\newcommand{\dC}{$^\circ$C}

\begin{document}

\title{Linear Optics in the PAW Formalism: A Small Addition to {\tt ABINIT}}

\author{J. W. Zwanziger (jzwanzig@dal.ca),
{\it Department of Chemistry and Institute for Research in Materials,
Dalhousie University, Halifax, NS B3H 4J3 Canada}}

\date{\today}

\maketitle

% main text
\section{Summary}

This document describes a modest extension of the {\tt conducti} program,
suitable for calculating linear optical response in the PAW formalism.
The original {\tt conducti} program, which has been part of {\tt abinit} for
some time, computes the optical conductivity in the Kubo-Greenwood formalism
using norm-conserving pseudopotentials. Recently, this functionality was
supplemented by the use of the PAW formalism, making it significantly
faster in many respects. In addition, the PAW formalism can be used to
compute the frequency dependent dielectric tensor in a similar way, and
indeed this is part of recent releases. However, the implementation up to
version 5.4 was coded such that only principle dielectric tensor values
were treated, therefore crystals of arbitrarily (low) symmetry were not
fully treated, and also there was no provision for a scissors correction. I
have re-written the linear optics calculation for {\tt conducti}, using
the PAW formalism as implemented already, but including provisions for a
scissors correction and all crystal symmetries.

\section{Computation}

Components of the imaginary part of the linear dielectric tensor are
computed as a function of energy from the perturbative formula\cite{Gajdos:06}
\begin{equation}
\label{baseform}
\epsilon_2^{ab}(\omega) = \frac{4\pi^2}{\Omega\omega^2}\sum_{n,m,\mathbf{k}} f_{nm}w_{\mathbf{k}}
\delta(\omega_{n,\mathbf{k}}-\omega_{m,\mathbf{k}}+\Delta-\omega)
\langle n_{\mathbf{k}}|-i\nabla_a|m_{\mathbf{k}}\rangle\langle n_{\mathbf{k}}|-i\nabla_b|m_{\mathbf{k}}\rangle^\ast
\end{equation}
where atomic units are assumed, and thus $\omega$ is an energy in Hartrees,
and all factors of $e$, $m$, and $\hbar$ are unity.  Superscripts $a$
and $b$ refer to cartesian directions, and the momentum operator is $-i\nabla_a =
-i \partial/\partial x_a$. Indices $n$ and $m$ refer to bands. The factor
$f_{nm}$ is the difference in occupation number between two bands. Summation over
the irreducible part of the Brillouin Zone uses
{\bf k} points. The Dirac delta function
in Eq.~\ref{baseform} is applied by replacing it with a Gaussian function, with
a width that acts to smear features of the spectrum and is an input parameter.
The factor $\Delta$ is the scissors correction,\cite{Levine:89} and is also an
input parameter. Finally, $\Omega$ is the unit cell volume.


After computation
of $\epsilon_2^{ab}$, the tensor is symmetrized, because in general the
integration is over only the irreducible part of the Brillouin Zone. The
tensor over the full Brillouin Zone is constructed as follows:
\begin{equation}
\epsilon_2^{ab}({\mathrm{BZ}}) = \frac{1}{\mathrm{nsym}} \sum_{\mathrm{i=1}}^{\mathrm{nsym}}S_i^{-1}\epsilon_2^{ab}({\mathrm{IBZ}})S_i,
\end{equation}
where the sum is over the {\tt nsym} point group symmetry elements $S$.

The real part of the dielectric tensor is computed from the imaginary part
by a Kramers-Kronig transform:
\begin{equation}
\epsilon_1^{ab}(\omega) = 1 + \frac{2}{\pi}{\cal P}\int d\omega' \frac{\omega' \epsilon_2^{ab}(\omega')}{\omega'^2-\omega^2}
\end{equation}
where ${\cal P}$ means to take the principal value.

\section{Usage}

First, do a normal ground state calculation, using PAW data (not
norm-conserving pseudopotentials) with {\tt prtwf 1} and {\tt prtnabla 1}
to save the wavefunction and momentum matrix elements. Next, prepare two files,
one that contains just the names of your input and output files, and the second (the
input file) that contains the relevant parameters. Thus for example I have one file
called {\tt con.files} that contains two lines: {\tt con.in} and {\tt con.out}. Then
my input file, {\tt con.in}, contains the following data:
\begin{verbatim}
3 # option for the program conducti, telling it to do linear optics calculation
sio # base name of files to read from ground state calculation
0.018 # scissors correction to apply, in Hartrees
0.009 0.0001 2.0 1000 # smearing width, minimum and maximum energies (all in Ha)
                      # and number of points to compute
\end{verbatim}
Then feed the file of file names to {\tt conducti}:
\begin{verbatim}
conducti < file.files
\end{verbatim}
The code will report some some details, and generate two files, {\tt con.out\_real} and
{\tt con.out\_imag} containing the real and imaginary parts of the dielectric tensor, as a
function of energy. These files have 7 columns, which are: energy, $\epsilon^{xx}$,
$\epsilon^{yy}$, $\epsilon^{zz}$, $\epsilon^{yz}$, $\epsilon^{xz}$, and $\epsilon^{xy}$.


\begin{thebibliography}{2}
\expandafter\ifx\csname natexlab\endcsname\relax\def\natexlab#1{#1}\fi
\expandafter\ifx\csname bibnamefont\endcsname\relax
  \def\bibnamefont#1{#1}\fi
\expandafter\ifx\csname bibfnamefont\endcsname\relax
  \def\bibfnamefont#1{#1}\fi
\expandafter\ifx\csname citenamefont\endcsname\relax
  \def\citenamefont#1{#1}\fi
\expandafter\ifx\csname url\endcsname\relax
  \def\url#1{\texttt{#1}}\fi
\expandafter\ifx\csname urlprefix\endcsname\relax\def\urlprefix{URL }\fi
\providecommand{\bibinfo}[2]{#2}
\providecommand{\eprint}[2][]{\url{#2}}

\bibitem[{\citenamefont{Gajdo\v{s} et~al.}(2006)\citenamefont{Gajdo\v{s},
  Hummer, Kresse, Furthm\"{u}ller, and Bechstedt}}]{Gajdos:06}
\bibinfo{author}{\bibfnamefont{M.}~\bibnamefont{Gajdo\v{s}}},
  \bibinfo{author}{\bibfnamefont{K.}~\bibnamefont{Hummer}},
  \bibinfo{author}{\bibfnamefont{G.}~\bibnamefont{Kresse}},
  \bibinfo{author}{\bibfnamefont{J.}~\bibnamefont{Furthm\"{u}ller}},
  \bibnamefont{and}
  \bibinfo{author}{\bibfnamefont{F.}~\bibnamefont{Bechstedt}},
  \bibinfo{journal}{Phys. Rev. B} \textbf{\bibinfo{volume}{73}},
  \bibinfo{pages}{045112} (\bibinfo{year}{2006}).

\bibitem[{\citenamefont{Levine and Allan}(1989)}]{Levine:89}
\bibinfo{author}{\bibfnamefont{Z.~H.} \bibnamefont{Levine}} \bibnamefont{and}
  \bibinfo{author}{\bibfnamefont{D.~C.} \bibnamefont{Allan}},
  \bibinfo{journal}{Phys. Rev. Lett.} \textbf{\bibinfo{volume}{63}},
  \bibinfo{pages}{1719} (\bibinfo{year}{1989}).

\end{thebibliography}
\end{document}

